\begin{otherlanguage}{english}

\begin{abstract}

% em inglês, o primeiro parágrafo não deve ser indentado
\noindent

Social networks have facilitated the sharing of information between people by providing an environment with low or no publishing barriers, where content created by a person can reach millions resulting in an overload of information that can confuse and/or mislead the user. In addition, people are increasingly using these environments to access news they once sought in specialized news institutions. Therefore, the greater the influence exerted by these platforms, the greater is the need to evaluate their content and to differentiate false and misleading information from truthful and informative ones. For this to be possible this paper proposes the use of a well-founded concept of social epistemology, which describes how people identify, recognize, and attribute authority to one another, known as the theory of cognitive authority. The hypothesis considered is that the foundations of the theory of cognitive authority can aid in the classification of news and the identification of false information in social networks. It is proposed a conceptual model to be implemented in the form of a software prototype, which considers an evaluation in a quantitative way, through the development of scenarios that will exemplify situations where the need to classify information and identify its quality occur. These scenarios will provide the dataset that will be used to validate or refute the hypothesis.

\end{abstract}

\end{otherlanguage}

