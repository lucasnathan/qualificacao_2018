\begin{resumo}

As redes sociais facilitaram o compartilhamento de informações entre as pessoas, fornecendo um ambiente com baixo ou nenhum filtro de publicação, onde o conteúdo criado por uma pessoa pode alcançar milhões, resultando em uma sobrecarga de informação, que pode confundir e/ou enganar os usuários da rede. Além disso, as pessoas estão cada vez mais utilizando estes ambientes para acessar notícias que outrora procuravam em noticiários especializadas. Portanto, quanto maior a influência exercida por essas plataformas sob as pessoas, maior a necessidade de avaliar seus conteúdos e diferenciar as informações falsas e enganosas das verdadeiras e informativas. Para que isso seja possível este trabalho propõe a utilização de um conceito fundamentado da epistemologia social, que descreve o modo como as pessoas identificam, reconhecem e atribuem autoridade umas às outras, conhecido como teoria da autoridade cognitiva. A hipótese considerada é de que os fundamentos da teoria da autoridade cognitiva podem auxiliar na classificação de notícias e na identificação de informação falsa em redes sociais. Propõe-se então, um modelo conceitual a ser implementado na forma de um protótipo de software, que considera uma avaliação de maneira quantitativa, através do desenvolvimento de cenários que irão exemplificar situações onde ocorrem a necessidade de classificar uma informação e identificar a sua qualidade. Estes cenários irão fornecer o conjunto de dados que será utilizado para validar ou refutar a hipótese.

\end{resumo}

